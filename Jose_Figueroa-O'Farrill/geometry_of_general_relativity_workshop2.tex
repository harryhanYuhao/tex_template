\documentclass[10pt,a4paper]{exam}
\usepackage[british]{babel}
% \usepackage{polyglossia}
% \setdefaultlanguage{english}
% \setotherlanguage{greek}

\usepackage[margin=2cm]{geometry}
\usepackage{amsmath,amssymb,amstext,amsthm,amscd,array,xcolor,pifont}
\usepackage{fontspec}
\defaultfontfeatures{Ligatures=TeX}
\setmainfont[
Path =/usr/share/texmf-dist/fonts/opentype/public/tex-gyre-math/,
Extension = .otf,
]{texgyrepagella-math.otf}
% \setmainfont[
% Path =/usr/share/texmf-dist/fonts/opentype/public/tex-gyre-math/,
% Extension = .otf,
% ]{TeX Gyre Pagella}
\usepackage[math-style=french]{unicode-math}
% \setmathfont{texgyrepagella-math.otf}
\usepackage{xunicode} % for unicode accents
%
\usepackage{mathrsfs}
%
\usepackage{tikz}
\usepackage{tikz-cd}
\usetikzlibrary{calc,arrows}
\usetikzlibrary{decorations.markings}
%
\usepackage{wrapfig}
\usepackage{subcaption,enumitem}
\usepackage{relsize}
%
%\usepackage{titlesec}
\usepackage{version,ifthen}
%
% this is for external references: it allows me to have \ref's whose
% \label's are in the Lecture notes
%
\usepackage{xr}
% \externaldocument{../Lectures/DGGR}
%
\usepackage[framemethod=TiKZ]{mdframed}
\makeatletter
\def\exam@MakeFramed#1{\begin{mdframed}}
\def\endexam@MakeFramed{\end{mdframed}}
\makeatother
%
\renewcommand{\subpartlabel}{(\thesubpart)}
%
\usepackage{hyperref}
\hypersetup{%
  pdftitle   = {GGR Workshop 2},
  pdfkeywords = {Conformal infinity},
  pdfauthor  = {José Figueroa-O'Farrill},
  pdfcreator = {\LaTeX\ with package ``hyperref''}
}
%
\definecolor{darkblue}{rgb}{0,0,0.6}
\definecolor{darkred}{rgb}{0.6,0,0}
\definecolor{darkgreen}{rgb}{0,0.6,0}
\definecolor{shadecolor}{rgb}{0.8,0,0.8}
%
\let\d\partial
\newcommand{\ttl}[1]{{\color{darkblue}\sffamily\bfseries\larger #1}}
\newcommand{\sttl}[1]{{\color{darkblue}\sffamily\bfseries #1}}
\newcommand{\doit}[1]{{\color{darkgreen}\sffamily\bfseries #1}}
\newcommand{\RR}{\mathbb{R}}
\newcommand{\FF}{\mathbb{F}}
\newcommand{\CC}{\mathbb{C}}
\newcommand{\VV}{\mathbb{V}}
\newcommand{\ZZ}{\mathbb{Z}}
\newcommand{\be}{\mathbf{e}}
\newcommand{\beff}{\mathbf{f}}
\newcommand{\bx}{\mathbf{x}}
\newcommand{\by}{\mathbf{y}}
\newcommand{\bu}{\mathbf{u}}
\newcommand{\eL}{\mathcal{L}}
\newcommand{\eX}{\mathcal{X}}
\newcommand{\eT}{\mathcal{T}}
\DeclareMathOperator{\Ort}{O}
\DeclareMathOperator{\End}{End}
\DeclareMathOperator{\Hom}{Hom}
\DeclareMathOperator{\GL}{GL}
\DeclareMathOperator{\SO}{SO}
\DeclareMathOperator{\grad}{grad}
\newcommand{\Ggr}{\mathcal{G}}
\newcommand{\V}{\mathscr{X}}
\newcommand{\F}{C^\infty}
\newcommand{\pder}[2]{\frac{\partial #1}{\partial #2}}
\newcommand{\der}[2]{\frac{d #1}{d #2}}
\newcommand{\scri}{\mathscr{I}}
\newcommand{\esu}{\mathscr{E}}
%
\pagestyle{headandfoot}
\firstpageheader{\ttl{GGR}}{\ttl{Workshop 2}}{\ttl{31/1/2025}}
\runningheader{}{}{}
\firstpagefooter{}{}{}
\runningfooter{}{}{}
%
\marksnotpoints
%
% answers?
%
% commenting out the line below will typeset the answers as well
\def\withanswers{1}
\ifdefined\withanswers
  \printanswers
\else
  \relax
\fi
%
\begin{document}

\begin{center}\large
  \textsc{Conformal infinity}
\end{center}

Let $(M,g)$ be a lorentzian manifold.  Let $\omega \in \F(M)$ and let
$\widehat g = e^{2\omega} g$.  Since $e^{2\omega}$ is a positive function,
$\widehat g$ is also a lorentzian metric.  We say that $g$ and $\widehat g$
are \sttl{conformally related}.  Let $\nabla$ and $\hat\nabla$ denote
the Levi-Civita connections of $g$ and $\widehat g$, respectively.
Let $K \in \Omega^1(M;\End(TM))$ be defined by
\begin{equation*}
  X \mapsto K_X \qquad\text{where}\qquad K_XY := \hat\nabla_X Y - \nabla_X Y,
\end{equation*}
for all $X,Y \in \V(M)$.

\begin{questions}

\question Following the steps below, show that
  \begin{equation}\label{eq:k}
    K_X Y = X(\omega) Y + Y(\omega) X - g(X,Y) \grad\omega,
  \end{equation}
  where the \sttl{gradient} $\grad\omega$ of $\omega$ is the metric dual
  of $d\omega \in \Omega^1(M)$; that is,
  \begin{equation*}
    g(\grad\omega, Z) = d\omega(Z) = Z(\omega) \qquad\text{for all}\qquad Z \in \V(M).
  \end{equation*}

  \begin{parts}
  \part Using that both $\hat\nabla$ and $\nabla$ are torsion-free,
    \doit{show} that
    \begin{equation}
      \label{eq:k-symm}
      K_X Y = K_Y X \qquad\text{for all $X,Y \in \V(M)$.}
    \end{equation}

    \begin{solution}
      We use that $\hat\nabla$ and $\nabla$ are torsion-free:
      \begin{align*}
        0 &= \hat\nabla_X Y - \hat\nabla_Y X - [X,Y] && \tag{since $\hat\nabla$ is torsion-free}\\
          &= \nabla_X Y - \nabla_Y X - [X,Y] + K_X Y - K_Y X && \tag{definition of $K$}\\
          &= K_X Y - K_Y Z, && \tag{since $\nabla$ is torsion-free}
      \end{align*}
      from where equation~\eqref{eq:k-symm} follows.
    \end{solution}

    \part Using that $\hat\nabla\widehat g= 0$ and $\nabla g = 0$, \doit{show}
      that
      \begin{equation}
        \label{eq:k-aux-xyz}
        g(K_X Y, Z) + g(Y, K_X Z) = 2 X(\omega) g(Y,Z) \qquad\text{for all
          $X,Y,Z \in \V(M)$.}
      \end{equation}
      
      \begin{solution}
        Now we use that $\hat\nabla \widehat g = 0$, so that for all
        $X,Y,Z \in \V(M)$,
        \begin{align*}
          0 = (\hat\nabla_X \widehat g)(Y,Z) &= X \widehat g(Y,Z) - \widehat g(\hat\nabla_X Y,Z ) - \widehat g(Y, \hat\nabla_X Z)\\
                                         &= X(e^{2\omega} g(Y,Z)) - e^{2\omega} g (\nabla_X Y + K_X Y, Z) - e^{2\omega} g (Y, \nabla_X Z + K_X Z)\\
                                         &= 2 e^{2\omega} X(\omega) g(Y,Z) +  e^{2\omega}\left( X g(Y,Z) - g(\nabla_X Y,Z) - g(Y,\nabla_X Z)\right)\\
                                         & \qquad {} - e^{2\omega} \left(g(K_X Y, Z) + g(Y, K_X Z) \right)\\
                                         &= e^{2\omega} \left(  2 X(\omega) g(Y,Z) - g(K_X Y, Z) - g(Y, K_X Z)\right), && \tag{since $\nabla g = 0$}
        \end{align*}
        from where equation~\eqref{eq:k-aux-xyz} follows.
      \end{solution}
      
    \part Using equations~\eqref{eq:k-symm} and \eqref{eq:k-aux-xyz},
      \doit{derive} equation~\eqref{eq:k}.\\
      (\emph{Hint}: Mimic the derivation of the Koszul formula for the
      Levi-Civita connection.)
      
      \begin{solution}
        We now take equation~\eqref{eq:k-aux-xyz} and we write two more
        equations by cyclically permuting $X,Y,Z$:
        \begin{equation}\label{eq:k-aux-yzx}
          g(K_Y Z, X) + g(Z, K_Y X) = 2 Y(\omega) g(Z,X)
        \end{equation}
        and
        \begin{equation}\label{eq:k-aux-zxy}
          g(K_Z X, Y) + g(X, K_Z Y) = 2 Z(\omega) g(X,Y).
        \end{equation}
        We now subtract equation~\eqref{eq:k-aux-zxy} from the sum of
        equations~\eqref{eq:k-aux-xyz} and \eqref{eq:k-aux-yzx}.
        Using equation~\eqref{eq:k-symm}, this results in
        \begin{align*}
          g(K_XY , Z) &= X(\omega) g(Y,Z) + Y(\omega) g(X,Z) - Z(\omega)  g(X,Y)\\
                      &= g( X(\omega) Y + Y(\omega) X - g(X,Y) \grad\omega, Z),
        \end{align*}
        which, since it is true for all $Z\in\V(M)$, is precisely
        equation~\eqref{eq:k}.
      \end{solution}
  \end{parts}

  \begin{EnvUplevel}
  \end{EnvUplevel}

\question Let $I = [0,1]$ and let $\gamma: I \to M$ be a smooth curve.
  Let $\gamma^*TM \to I$ be the pull-back bundle, whose sections we
  interpret as ``vector fields along $\gamma$''.  Let $\frac{\widehat
    D}{dt}$ and $\frac{D}{dt}$ denote the covariant derivatives acting
  on vector fields along $\gamma$, associated with $\hat\nabla$ and
  $\nabla$, respectively.

  \begin{parts}
  \part Let $Y \in \Gamma(\gamma^*TM)$.  \doit{Show} that
    \begin{equation*}
      \frac{\widehat D Y}{dt} = \frac{D Y}{dt} + K_{\dot\gamma} Y,
    \end{equation*}
    where we are abusing notation slightly and writing
    $K_{\dot\gamma}$ even when $\dot \gamma$ is not a vector field on
    $M$, but $K$ is a tensor and hence it can be restricted to a
    curve.\\
    (\emph{Hint:} It is enough to prove this for $Y = f \gamma^*Z$ for
    some $f \in \F(I)$ and $Z \in \V(M)$. Why?)

    \begin{solution}
      By linearity of the covariant derivative and the fact that locally every section of $\gamma^*TM$ is a linear combination (with coefficients in $\F(I)$) of pull-backs of sections of $TM$, it is enough to prove the identity for $Y = f \gamma^*Z$ as in the hint.
      So let $f \in \F(I)$, $Z \in \V(M)$ and consider $Y = f \gamma^*Z$.  Calculating
      \begin{align*}
        \frac{\widehat D}{dt}(f \gamma^* Z) &= \der{f}{t} \gamma^*Z + f \gamma^*(\hat\nabla_{\dot\gamma} Z)\\
                                             &= \der{f}{t} \gamma^*Z + f \gamma^*(\nabla_{\dot\gamma} Z) + f \gamma^*(K_{\dot\gamma}Z)\\
                                             &= \frac{D}{dt} (f \gamma^* Z) + f \gamma^*(K_{\dot\gamma} Z) \\
                                             &= \frac{D}{dt} (f \gamma^*Z) + K_{\dot\gamma} (f \gamma^* Z).
      \end{align*}
    \end{solution}

    \part \doit{Show} that if $\gamma : I \to M$ is a null geodesic with
      respect to $g$, it is also a null geodesic with respect to $\widehat
      g$. (These geodesics need not be affinely parametrised and
      recall that $\gamma$ being null with respect to a metric $g$
      means that $g(\dot\gamma,\dot\gamma) = 0$.)

      \begin{solution}
        First of all $\widehat g (\dot\gamma,\dot\gamma) = e^{2\omega}
        g(\dot\gamma,\dot\gamma)$, so that if $\dot\gamma$ is null
        relative to $g$ it is also null relative to $\widehat g$ (and
        viceversa).  Now let us calculate
        \begin{align*}
          \frac{\widehat D \dot\gamma}{dt} &= \frac{D\dot\gamma}{dt} +  K_{\dot\gamma} \dot\gamma && \tag{from part (a)}\\
                                       &= f \dot\gamma +  K_{\dot\gamma} \dot\gamma && \tag{$\exists f \in \F(I)$, since $\gamma$ is a $\nabla$-geodesic}\\
                                       &= f \dot\gamma + 2 \dot\gamma(\omega) \dot\gamma && \tag{by \eqref{eq:k} and $g(\dot\gamma,\dot\gamma)=0$}\\
                                       &= (f + 2 \der{~}{t}(\omega \circ \gamma)) \dot\gamma, && \tag{since $\dot\gamma = \gamma_*(\der{~}{t})$}
        \end{align*}
        which shows that $\gamma$ is a $\hat\nabla$-geodesic; although not necessarily affinely parametrised.
      \end{solution}
  \end{parts}

\question Now let $(M, g)$ denote four-dimensional Minkowski
  spacetime.  Relative to the global cartesian coordinates
  $(t,x,y,z)$, the metric takes the form
  \begin{equation*}
    g = -dt^2 + dx^2 + dy^2 + dz^2.
  \end{equation*}
  Notice that $t$ here is one of the global coordinates of Minkowski
  spacetime and not the parameter along a curve.
  
  Let us introduce spherical polar coordinates $(r,\theta,\phi)$ with
  \begin{equation*}
      x = r \sin\theta\cos\phi, \qquad y = r \sin\theta\sin\phi \qquad\text{and}\qquad  z = r \cos\theta,
  \end{equation*}
  relative to which the metric takes the form
  \begin{equation}\label{eq:mink-polar}
    g = -dt^2 + dr^2 + r^2 (d\theta^2 + \sin^2\theta d\phi^2),
  \end{equation}
  where $t \in \RR$ and $r\geq 0$.  (Strictly speaking,
  the coordinate system breaks down whenever $r=0$ or $\phi = 0$.)  We will use the
  shorthand $h = d\theta^2 + \sin^2\theta d\phi^2$ for the round
  metric on the unit sphere in $\RR^3$, since the sphere will not play
  a huge rôle in the discussion.

  \begin{parts}
  \part Define \sttl{retarded} $u=t-r$ and \sttl{advanced} $v=t+r$
    coordinates.  Since $r \geq 0$, we have that $u,v \in \RR$ with $v
    \geq u$.  \doit{Rewrite} the Minkowski metric
    \eqref{eq:mink-polar} in these coordinates.

    \begin{solution}
      This is a simple coordinate substitution and we obtain
      \begin{equation*}
        - du dv + \tfrac14 (v-u)^2 h, \qquad\text{with $v \geq u$.}
      \end{equation*}
    \end{solution}

    \part Now let us ``compactify'' the range of the $u,v$ coordinates
      by $u= \tan U$ and $v = \tan V$, where now $U,V \in
      (-\tfrac\pi2,\tfrac\pi2)$.  \doit{Show} that the Minkowski metric in
      these coordinates is given by
      \begin{equation*}
        g = \frac{1}{4 \cos^2 U \cos^2 V} \left(-4 dU dV +
          \sin^2(V-U) h \right), \qquad\text{with $V \geq U$.}
      \end{equation*}

      \begin{solution}
        This is again just a simple coordinate substitution.  The only
        tricky thing is the range of coordinates.  Notice that, since
        $\der{\tan x}{x} = 1 + \tan^2 x \geq 1$, $\tan$ is a
        monotonically increasing function, hence $\tan V \geq \tan U$
        if and only if $V \geq U$.
      \end{solution}

      \begin{EnvUplevel}
        Now we define the conformally related metric $\hat g =
        e^{2\omega} g$, where $e^{\omega} = 2\cos U \cos V$:
        \begin{equation*}
          \hat g  = -4 dU dV + \sin^2(V-U) h, \qquad\text{with $V \geq U$.}
        \end{equation*}
        Notice that whereas the metric $g$ is singular if either $U$
        or $V$ equals $\pm\frac\pi2$, the metric $\hat g$ is regular
        there.
      \end{EnvUplevel}

    \part Change coordinates from $(U,V)$ to $(R,T)$, where $T= V+U
      \in (-\pi,\pi)$ and $R = V-U \in (0,\pi)$ and \doit{show} that the
      metric $\hat g$ becomes
      \begin{equation*}
        \hat g = -dT^2 + dR^2 + \sin^2R\ h.
      \end{equation*}
      Notice that the metric extends smoothly to all real values of $T$.

      \begin{solution}
        This is again just a simple coordinate substitution and the
        ranges of $R,T$ are clear.
      \end{solution}

    \part \doit{Show} that the spatial metric $dR^2 + \sin^2 R \ h$ is the
      round metric on the three-dimensional unit sphere $S^3 \subset
      \RR^4$.\\
      (\emph{Hint}: The round metric on the unit sphere is obtained by
      pulling back the euclidean metric on $\RR^4$.  Compare with
      Example~\ref{xm:round-metric-s2} in the lecture notes.)

      \begin{solution}
        In the $(R,\theta,\phi)$ coordinate system, the spatial
        metric is
        \begin{equation*}
          dR^2 + \sin^2 R d\theta^2 + \sin^2R \sin^2\theta
          d\phi^2,
        \end{equation*}
        which is the pull-back of the euclidean metric on $\RR^4$
        \begin{equation*}
          \sum_{i=1}^4 (dx^i)^2
        \end{equation*}
        under the embedding
        \begin{equation*}
          \begin{split}
            x^1(R,\theta,\phi) &= \sin R \sin\theta \cos\phi\\
            x^2(R,\theta,\phi) &= \sin R \sin\theta \sin\phi\\
            x^3(R,\theta,\phi) &= \sin R \cos\theta\\
            x^4(R,\theta,\phi) &= \cos R.
          \end{split}
        \end{equation*}
        Notice that $\sum_{i=1} (x^i)^2 = 1$, so that the image of the
        embedding is (most of) the unit sphere in $\RR^4$.
      \end{solution}

      \begin{EnvUplevel}
        This shows that Minkowski spacetime embeds conformally into
        $\RR \times S^3$, which is called the \sttl{Einstein static
          universe} $\esu$.
      \end{EnvUplevel}
    \end{parts}

  \question The following is a diagram in the $(R,T)$ plane of the
        image of Minkowski spacetime under the conformal embedding
        $(M,g) \to (\esu, \hat g)$.  (It is called the \sttl{Penrose
          diagram} of Minkowski spacetime.)  We have added the
        \sttl{conformal boundary} consisting of the following subsets of $\esu$:

        \begin{minipage}[t]{0.60\textwidth}
          \begin{itemize}
          \item \sttl{future timelike infinity} $i^+$, which is $(R,T)=(0,\pi)$, or $U=\frac\pi2$ and $V=\frac\pi2$;
          \item \sttl{future null infinity} $\scri^+$, which is $R+T=\pi$, or $V=\frac\pi2$ and $U\in(-\frac\pi2,\frac\pi2)$;
          \item \sttl{spacelike infinity} $i^0$, which is $(R,T)=(\pi,0)$, or $U=-\frac\pi2$ and $V=\frac\pi2$; 
          \item \sttl{past null infinity} $\scri^-$, which is $R-T=\pi$, or $U=-\frac\pi2$ and
            $V\in(-\frac\pi2,\frac\pi2)$; and
          \item \sttl{past timelike infinity} $i^-$, which is $(R,T)=(0,-\pi)$, or $U=-\frac\pi2$ and $V=-\frac\pi2$.
          \end{itemize}
        \end{minipage}
        \begin{minipage}[t]{0.39\textwidth}
          \centering
          \begin{tikzpicture}[x=1.0cm,y=1.0cm,scale=0.7,baseline={(0,2.75)}]
            \coordinate [label=above right:{$i^+$}] (ip) at (0,2); 
            \coordinate [label=below right:{$i^-$}] (im) at (0,-2);
            \coordinate [label=above right:{$i^0$}] (i0) at (2,0);
            \draw [->,thin,color=gray] (-1,0) -- (3,0) node [black,below right] {$R$};
            \draw [->,thin,color=gray] (0,-3) -- (0,3) node [black,above left] {$T$};
            \fill [blue,opacity=0.1] (im) -- (ip) -- (i0) -- cycle;
            \draw [line width=2pt, color=black] (im) -- (i0) node [midway, right] {$\scri^-$};
            \draw [line width=2pt, color=black] (i0) -- (ip) node [midway, above right] {$\scri^+$};
            \foreach \point in {ip,im,i0}
            \filldraw [color=blue!70!black,fill=blue!50!white] (\point) circle (2pt);
          \end{tikzpicture}
        \end{minipage}

        \begin{parts}
        \part \doit{Show} that whereas $i^0$ and $i^\pm$ are points of
          $\esu$, $\scri^\pm \subset \esu$ are \sttl{null
            hypersurfaces} -- i.e.,  three-dimensional submanifolds on
          which the metric $\hat g$ degenerates.

          \begin{solution}
            The coordinate $R \in [0,\pi]$ is related to the latitude
            on the $3$-sphere: $R=0$ is the North pole and $R=\pi$ is
            the South pole.  The subsets of the $3$-sphere with
            constant $R \in (0,\pi)$ are $2$-spheres of radius $\sin
            R$, so that for $R=0,\pi$ they are $2$-spheres of zero
            radius, i.e., points.  This explains why $i^\pm$ and $i^0$
            are points.

            The points on $\scri^\pm$ with fixed values of $(R,T)$ are
            actually $2$-spheres with radius $\sin R$.  As $(R,T)$
            vary in $\scri^\pm$ we get a one-parameter family of
            $2$-spheres of radii $\sin(\pm T)$ for $\pm T \in (0,\pi)$.
            Therefore they are hypersurfaces (i.e., codimension-one
            submanifolds) of $\esu$.

            To work out the metric $\hat g$ on $\scri^\pm$, we
            parametrise $\scri^\pm$ by $(T,\theta,\phi)$, where $R=\pi
            \mp T$.  We see that $dR = \mp dT$ and hence $-dT^2 + dR^2
            = 0$.  The metric $\hat g$ then reduces to
            \begin{equation*}
             \left.\hat g \right|_{\scri^\pm} = \sin^2 T (d\theta^2 + \sin^2\theta d\phi^2),
            \end{equation*}
            which is clearly degenerate since, e.g., $\pder{~}{T}$ is
            orthogonal to every tangent vector.
          \end{solution}

        \part \doit{Show} that null geodesics of $(M,g)$ end at
          $\scri^\pm$.  You may restrict yourself to ``radial'' null
          geodesics, for which the spherical coordinates $\theta,\phi$
          are constant.  Also recall that in Question 2(b) you showed
          that conformally related metrics share the same null
          geodesics.

          \begin{solution}
            A radial null geodesic in $(M,g)$ is
            a null geodesic in $(\esu,\hat g)$ which only involves the
            $(R,T)$ coordinates, so that $\dot\gamma$ lives in the
            span of $\pder{~}{R}$ and $\pder{~}{T}$.  Being null,
            $\dot\gamma$ is proportional to $\pder{~}{R} \pm
            \pder{~}{T}$, so the image of $\gamma$ is a straight line
            segment at 45\textdegree\ to the $T$ axis.  If they
            are future-pointing (i.e., if the coefficient of
            $\pder{~}{T}$ is positive), they will eventually hit
            $\scri^+$, whereas if they are past-pointing (i.e., if the
            coefficient of $\pder{~}{T}$ is negative), they will
            eventually hit $\scri^-$.  In summary, any radial null
            geodesic hits either of $\scri^\pm$.  These hypersurfaces
            are at ``infinity'' in $(M,g)$ and will take an infinite
            amount of affine time to get there.  The conformal
            embedding of $(M,g)$ into $(\esu,\hat g)$ brings them to a
            finite ``distance''.
          \end{solution}

          
        \part \doit{Show} that $i^0$ is the endpoint of spacelike
          geodesics in $(M,g)$ and \doit{show} that $i^\pm$ are the
          endpoints of timelike geodesics in $(M,g)$.  Notice that
          $(M,g)$ and $(\esu,\hat g)$ do not have the same non-null
          geodesics, so you cannot just argue by looking at geodesics
          in $(\esu,\hat g)$.  On the other hand, symmetry
          considerations guarantee that it is enough to consider the
          timelike geodesic with $\dot\gamma = \pder{~}{t}$ and the
          spacelike geodesic with $\dot\gamma = \pder{~}{z}$, say.
        \end{parts}

        \begin{solution}
          Let $\lambda$ denote the affine parameter of the geodesic
          and let $\dot\gamma = \gamma_*(\der{~}{\lambda})$.  For the
          timelike geodesic with $\dot\gamma = \pder{~}{t}$, we have
          that the geodesic has coordinates $(t_0+ \lambda, r_0,
          \theta_0, \phi_0)$, so that $u(\lambda) = \lambda + t_0 -
          r_0$ and $v(\lambda) = \lambda + t_0 + r_0$.  In the limit
          $\lambda \to \infty$, $u,v \to \infty$ so that $U,V \to
          \tfrac\pi2$.  In other words, $\gamma \to i^+$.  Similarly,
          in the limit $\lambda \to -\infty$, $\gamma \to i^-$.

          The geodesic with $\dot\gamma = \pder{~}{z}$ has coordinates
          $(t_0, x_0, y_0, z_0 + \lambda)$.  Then
          \begin{equation*}
            \begin{split}
              u(\lambda) &= t_0 - \sqrt{x_0^2 + y_0^2 + (z_0 + \lambda)^2}\\
              v(\lambda) &= t_0 + \sqrt{x_0^2 + y_0^2 + (z_0 + \lambda)^2},
            \end{split}
          \end{equation*}
          so as $\lambda \to \pm \infty$, we have that
          $u \to -\infty$ and $v \to \infty$, or equivalently, $(U,V)
          \to (-\frac\pi2,\frac\pi2)$.  In other words, $\gamma \to
          i^0$.
        \end{solution}

      \begin{EnvUplevel}
        In this example, the Minkowski metric $g$ is known as the
        \sttl{physical metric} in the GR literature, whereas the
        metric $\hat g$ in $\esu$ is known as the \sttl{unphysical
          metric}.  The metric $\hat g$ is not unique, since we could
        have used a different function $\omega$ to rescale $g$.
        Nevertheless one can show that the conformal boundary is
        intrinsic to $(M,g)$.
      \end{EnvUplevel}
\end{questions}

\end{document}


% Local Variables:
% coding: utf-8
% mode: LaTeX
% TeX-command-default: "XeLaTeX"
% TeX-engine: "xetex"
% TeX-master: t
% End:

